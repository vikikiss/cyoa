\documentclass[12pt,a4paper,oneside]{report}
\usepackage{txfonts}
\usepackage[utf8]{inputenc}
\usepackage[T1]{fontenc}
\usepackage[magyar]{babel}
\usepackage{amssymb}
\usepackage{paralist}
\begin{document}
\title{Kaland játék kockázat}
\author{Kiss Viktória}
\date{2011.\ június}
\maketitle
\tableofcontents

\addcontentsline{toc}{chapter}{Bevezetés}
\chapter*{Bevezetés}
  A nagyprogramom alapjául a '80-'90-es évek \emph{Kaland Játék Kockázat}
  könyvei szolgáltak. Ezekben a könyvekben nem hagyományos történetek
  szerepelnek, tehát nem az előre megírt kalandot, történetet olvassa el
  az olvasó, helyette sok apró fejezetre van bontva a cselekmény, melyek
  végén választási lehetőség áll. Így az olvasóra van bízva, hogy hogyan
  folytatja tovább a történetet. Vagyis az olvasó lett maga a kalandor,
  aki kedve szerint formálhatta a történéseket attól függően, hogy
  mennyire bátor, kíváncsi, leleményes és kalandvágyó. Ezek a kalandok
  nagyrészt nem valós történeteket foglalnak magukba, hanem a fantázia
  világában játszódnak, meseszerű fordulatokkal, mesebeli lényekkel. Az
  olvasónak, vagyis a történet hősének így óriásokkal, szörnyekkel kell
  megvívni, hogy elérje a kitűzött célt, melyet a kaland elején kap.
  
  Ám sajnos a kalandot el is lehet bukni. Vannak olyan helyek, ahol
  csapdába eshetünk, vagy egy gonosz lény felül kerekedik rajtunk, és
  nem tudjuk legyőzni. Ahhoz, hogy sikeresen végigjátszuk szükségünk
  van \textsc{ügyességre}, \textsc{szerencsére} és
  \textsc{életerőre}. Ez a három dolog, ami jellemez egy hőst.

  Mivel a játékos mondja meg, hogy mi legyen a következő lépés, ő
  befolyásolja a további történést, ezért jó ötletnek találtam ezt
  megvalósítani egy programozási nyelven. Minden oldal elején a történet
  folytatódik és az oldal végeztével a játékos megmondhatja, hogy
  melyik kínálkozó utat választja. Így folyamatos interakcióra van
  szükség a játékos részéről. Miután eldöntötte, hogy merre folytatja
  kalandját, átmegyünk arra az oldalra, ahol további kalandok, veszélyek
  leselkednek rá.
  
  A játékoz szükség volt egy papírlapra, melyen a játékos vezette, hogy
  mennyi aktuálisan az életereje, szerencséje, ügyessége. Illetve, ha
  talált egy tárgyat, amit fel lehet venni, azt is felírta, hogy
  használni tudja később, amikor szükség van rá. Emellett két dobókocka
  is kellett, mert a szörnyek elleni harcnál a véletlen dönt, hogy ki
  sebez kit. A nagyprogramom végigjátszásánál nincs szükség ezekre a
  kellékekre, mert a program állapota tárolja, hogy milyen eszközöket vettünk
  fel, mennyi az életerő, szerencse és ügyességi pontjaink, mennyit
  dobtunk a dobokockával, stb.
  
\chapter{Felhasználói dokumentáció}

    \section{Játékszabályok}
  
      \subsection{Ügyesség, életerő, szerencse pontok}
        Mielőtt elkezdünk egy játékot meg kell határoznunk a fent említett
        életerő, szerencse és ügyességi pontjainkat. Ahhoz, hogy megkapjuk az
        ügyességi pontainkat dobjunk a dobókockával, majd a kapott számhoz
        adjunk hozzá 6-ot. Ugyanígy járjunk el a szerencsepontjaink
        meghatározásához is. Hogy életerő pontjainkat megtudjuk dobjunk két
        dobókockával és adjunk hozzá 12-t. Ezek az értékek a kalandok során
        folyamatosan változni fognak, de nem csökkenhetnek 0 alá, és pont
        visszaszerzés során nem mehetnek a kezdeti érték fölé. Az ügyességi
        pontok mutatják, hogy mennyire tudunk jól harcolni, mennyire vagyunk
        tapasztaltak. Az életerő pont azt jelzi, mennyire vagyunk
        egészségesek, erősek. A szerencse, pedig természetesen a szerencsénket
        jelzi.
    
      \subsection{Csata}
        Sűrűn találkozhatunk kalandunk során olyan oldallal, ahol meg kell
        küzdenünk egy különleges teremtménnyel, mesebeli szörnnyel. A szörny is
        rendelkezik hasonló tulajdonságokkal, mint mi, vagyis neki is van
        ügyességi és életerő pontja, viszont nincs szerencse pontja.
        Amikor elérkezünk a csatához, akkor körökre bontva támadunk. Vagy a
        lény támad minket, vagy mi a lényt. Ennek eldöntésére használjuk a
        dobókockát, hogy véletlenszerű legyen. Ezeket a köröket addig
        csináljuk, amíg valakinek az életereje le nem csökken 0-ra. Ha a
        szörny életereje lemegy 0-ra, akkor mi győztünk, ellenkező esetben
        vesztettünk és újból kell kezdenünk az egész kalandunkat, nem csak a
        harcot. Egy kör az alábbi módon néz ki:
        \begin{enumerate}
          \item Dobjunk két kockával a szörny nevében, és két kockadobás összegét adjuk
            hozzá a szörny ügyességi pontjaihoz. Ez lesz az ő \textsc{támadóereje} ebben a körben.
          \item Dobjunk két kockával a saját nevünkben és hasonlóan, mint az előbb,
            ezt az összeget adjuk hozzá a mi ügyességi pontjainkhoz. Ez a saját
            \textsc{támadóerőnk}.
          \item Ekkor össze kell hasonlítani a kapott két számot.
            \begin{itemize}
              \item Abban az esetben ha megegyeznek, akkor kivédtük egymás
                támadását. Nem sérült senki, nem kerekedett felül senki. Kezdhetjük
                elölről a kört. Minden marad ugyanúgy.
              \item Ha a mi támadóerőnk a nagyobb, akkor mi sebeztük a szörnyet. Így a
                szörny életerejéből levonunk 2-t.
              \item Ha az ő támadóereje a nagyobb, akkor sebzést kaptunk, így a mi
                életerőnk csökken 2-vel.
              \item Viszont ezek után még lehetőségünk van javítani az
                eredményen. Próbára tehetjük a szerencsénket, ami azt jelenti, hogy
                ismét dobnunk kell mindkét dobókockával. Ha a kapott számok összege
                kisebb vagy egyenlő, mint a jelenlegi szerencsepontjaink száma,
                akkor vagyunk szerencsések, ellenkező esetben
                balszerencsések.
                \begin{compactitem}[--]
                  \item Ha mi sebeztünk és szerencsénk volt, az azt jelenti, hogy további sebzést
                    tudtunk bevinni a szörnynek, így még 2 pontot levonhatunk tőle.
                  \item Ha mi sebződtünk és szerencsétlenek voltunk, akkor a szörnynek sikerült
                    még egy csapást bevinni, ezért mi még 2 életerő pontot vesztünk.
                  \item Ha mi sebeztünk, de szerencsétlenül jártunk, akkor egy pontot
                    visszakap az ellenfél.
                  \item Ha sebzést kaptunk és szerencsétlenek voltunk, akkor még egy
                    pontot elveszítünk.
                \end{compactitem} 
            \end{itemize}
            Minden esetben, amikor igénybe vesszük a
            szerencsénket, akkor a szerencsepontjaink száma egyel csökken.
          \item A kör végén ellenőriznünk kell, hogy nem csökkent-e le a mi vagy az
            ellenfelünk életerő pontja. Abban az esetben ha igen, akkor vagy
            legyőztük a szörnyet, és folytathatjuk tovább kalandunkat, vagy ha
            az ellenfél győzőtt le minket, akkor elölről kell kezdeni az egész
            kalandot az első oldaltól.
        \end{enumerate} 
      \subsection{Szerencse}
        Nemcsak csatában találkozhatunk azzal a kifejezéssel, hogy
        \emph{Tedd próbára a szerencsédet!}. Ekkor nem mi döntünk sorsunk
        alakulásáról, hanem a szerencse befolyásolja kalandunk további
        menetét. Hasonlóan, mint a csatában dobnunk kell a két kockával
        egyszerre. Ha az összegük kevesebb vagy egyenlő, mint a szerencse
        pontjaink száma, akkor szerencsések vagyunk. Ha azonban nagyobb, akkor
        szerencsétlenek vagyunk. Ennek fényében kell tovább folytatni a
        kalandunkat. Minden szerencsepróba után a szerencse pontjaink száma
        egyel csökken.
    
      \section{Játék menete}
        A kaland az 1. oldalon kezdődik és minden oldal végén választani kell,
        hogy merre szeretnénk tovább haladni. Van, hogy csak két út áll
        előttünk, de van, hogy sok lehetőség közül választhatunk. Cél, hogy
        eljussunk a 400. oldalra (ahol a győzelem vár ránk), melynek eléréséhez
        különböző megmérettetések, harcok, próbák, észjátékok várnak ránk.
        Először kocka dobásokkal meghatározzuk a tulajdonságainkat, majd
        elolvashatjuk a háttértörténetet, amelyből kiderül, hogy a hősnek,
        akinek kalandjait mi választjuk meg, mi a célja. Honnan indul és
        hová akar érkezni.
    \section{Program indítása, használata}
      TODO
\chapter{Fejlesztői dokumentáció}
\end{document}

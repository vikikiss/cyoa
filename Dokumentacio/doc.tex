\documentclass[12pt,a4paper,oneside]{report}

\usepackage{txfonts}
\usepackage[utf8]{inputenc}
\usepackage[T1]{fontenc}
\usepackage[magyar]{babel}
\usepackage{amssymb}
\usepackage{paralist}
\usepackage{graphicx}
\usepackage[usenames,dvipsnames]{color}

\definecolor{Tag}{named}{NavyBlue}
\definecolor{Attr}{RGB}{238,0,0}
\definecolor{Value}{named}{OliveGreen}

\newcommand{\stat}{\textsc}
\newcommand{\attr}{\emph}
\newcommand{\proglang}{\texttt}

\begin{document}
\begin{titlepage}

\begin{tabular}{p{1in}p{0.8in}r}
\parbox{3cm}{\includegraphics[height=30mm]{eltecimer}} & & 
\parbox{7.0cm}{
  Eötvös Loránd Tudományegyetem \\
  Informatikai Kar}
\\[6.0cm]
\end{tabular}

\begin{center}
{\Huge Kaland -- Játék -- Kockázat }\\ [0.3cm]
{\Large Choose Your Own Adventure} \\ [1cm]
{\Large Kiss Viktória} \\ [0.3cm]
{\Large KIVMAAT.ELTE} \\ [0.3cm]
{\Large Programtervező matematikus} \\ [0.3cm]
{\Large Nappali tagozat} \\ [1cm]
{\Large Konzulens: Dr. Kozsik Tamás}
\end{center}
\vfill{\large Budapest, 2011}

\end{titlepage}

\tableofcontents

\newcommand{\chap}[1]{
  \addcontentsline{toc}{chapter}{#1}
  \addtocounter{chapter}{1}
  \chapter*{#1}
}

\addcontentsline{toc}{chapter}{Bevezető}
\chapter*{Bevezetés}

  A nagyprogramom alapjául a '80-'90-es évek \emph{Kaland Játék Kockázat}
  könyvei szolgáltak. Ezekben a könyvekben nem hagyományos történetek
  szerepelnek, tehát nem az előre megírt kalandot, történetet olvassa el
  az olvasó, helyette sok apró fejezetre van bontva a cselekmény, melyek
  végén választási lehetőség áll. Így az olvasóra van bízva, hogy hogyan
  folytatja tovább a történetet. Vagyis az olvasó lett maga a kalandor,
  aki kedve szerint formálhatta a történéseket attól függően, hogy
  mennyire bátor, kíváncsi, leleményes és kalandvágyó. Ezek a kalandok
  nagyrészt nem valós történeteket foglalnak magukba, hanem a fantázia
  világában játszódnak, meseszerű fordulatokkal, mesebeli lényekkel. Az
  olvasónak, vagyis a történet hősének így óriásokkal, szörnyekkel kell
  megvívni, hogy elérje a kitűzött célt, melyet a kaland elején kap.
  
  Ám sajnos a kalandot el is lehet bukni. Vannak olyan helyek, ahol
  csapdába eshetünk, vagy egy gonosz lény felül kerekedik rajtunk, és
  nem tudjuk legyőzni. Ahhoz, hogy sikeresen végigjátszuk szükségünk
  van \stat{ügyességre}, \stat{szerencsére} és \stat{életerőre}. Ez a
  három dolog, ami jellemez egy hőst.

  Mivel a játékos mondja meg, hogy mi legyen a következő lépés, ő
  befolyásolja a további történést, ezért jó ötletnek találtam ezt
  megvalósítani egy programozási nyelven. Minden oldal elején a történet
  folytatódik és az oldal végeztével a játékos megmondhatja, hogy
  melyik kínálkozó utat választja. Így folyamatos interakcióra van
  szükség a játékos részéről. Miután eldöntötte, hogy merre folytatja
  kalandját, átmegyünk arra az oldalra, ahol további kalandok, veszélyek
  leselkednek rá.
  
  Az eredeti játék az volt, hogy a könyvet olvasva a játékos
  kigyűtötte magának az információkat, amelyet kalandja során
  tapasztalt így, szükség volt egy papírlapra, melyen vezette, hogy
  mennyi aktuálisan az \stat{életereje}, \stat{szerencséje},
  \stat{ügyessége}. Illetve, ha talált egy tárgyat, amit fel lehet
  venni, azt is felírta, hogy használni tudja később, amikor szükség
  van rá. Emellett két dobókocka is kellett, mert a szörnyek elleni
  harcnál a véletlen dönt, hogy ki sebez kit. A nagyprogramom
  végigjátszásánál nincs szükség ezekre a kellékekre, mert a program
  állapota tárolja, hogy milyen eszközöket vettünk fel, mennyi az
  \stat{életerő}, \stat{szerencse} és \stat{ügyességi pontjaink},
  mennyit dobtunk a dobokockával, stb.
  
\chap{Felhasználói dokumentáció}

    \section{Játékszabályok}
  
      \subsection{Ügyesség, életerő, szerencse pontok}
        Mielőtt elkezdünk egy játékot meg kell határoznunk a fent
        említett \stat{életerő}, \stat{szerencse} és \stat{ügyességi
        pontjainkat}. Ahhoz, hogy megkapjuk az \stat{ügyességi
        pontainkat} dobjunk a dobókockával, majd a kapott számhoz
        adjunk hozzá 6-ot. Ugyanígy járjunk el a \stat{szerencse
        pontjaink} meghatározásához is. Hogy \stat{életerő
        pontjainkat} megtudjuk dobjunk két dobókockával és adjunk
        hozzá 12-t. Ezek az értékek a kalandok során folyamatosan
        változni fognak, de nem csökkenhetnek 0 alá, és pont
        visszaszerzés során nem mehetnek a kezdeti érték fölé. Az
        \stat{ügyességi pontok} mutatják, hogy mennyire tudunk jól
        harcolni, mennyire vagyunk tapasztaltak. Az \stat{életerő
        pont} azt jelzi, mennyire vagyunk egészségesek, erősek. A
        \stat{szerencse}, pedig természetesen a szerencsénket jelzi.
    
      \subsection{Csata}
        Sűrűn találkozhatunk kalandunk során olyan oldallal, ahol meg
        kell küzdenünk egy különleges teremtménnyel, mesebeli
        szörnnyel. A szörny is rendelkezik hasonló tulajdonságokkal,
        mint mi, vagyis neki is van \stat{ügyességi} és \stat{életerő
        pontja}, viszont nincs \stat{szerencse pontja}.  Amikor
        elérkezünk a csatához, akkor körökre bontva támadunk. Vagy a
        lény támad minket, vagy mi a lényt. Ennek eldöntésére
        használjuk a dobókockát, hogy véletlenszerű legyen. Ezeket a
        köröket addig csináljuk, amíg valakinek az életereje le nem
        csökken 0-ra. Ha a szörny életereje lemegy 0-ra, akkor mi
        győztünk, ellenkező esetben vesztettünk és újból kell
        kezdenünk az egész kalandunkat, nem csak a harcot. Egy kör az
        alábbi módon néz ki:
        \begin{enumerate}
          \item Dobjunk két kockával a szörny nevében, és két kockadobás összegét adjuk
            hozzá a szörny ügyességi pontjaihoz. Ez lesz az ő \textsc{támadóereje} ebben a körben.
          \item Dobjunk két kockával a saját nevünkben és hasonlóan, mint az előbb,
            ezt az összeget adjuk hozzá a mi ügyességi pontjainkhoz. Ez a saját
            \textsc{támadóerőnk}.
          \item Ekkor össze kell hasonlítani a kapott két számot.
            \begin{itemize}
              \item Abban az esetben ha megegyeznek, akkor kivédtük egymás
                támadását. Nem sérült senki, nem kerekedett felül senki. Kezdhetjük
                elölről a kört. Minden marad ugyanúgy.
              \item Ha a mi \textsc{támadóerőnk} a nagyobb, akkor mi sebeztük a szörnyet. Így a
                szörny \stat{életerejéből} levonunk 2-t.
              \item Ha az ő \textsc{támadóereje} a nagyobb, akkor sebzést kaptunk, így a mi
                \stat{életerőnk} csökken 2-vel.
              \item Viszont ezek után még lehetőségünk van javítani az
                eredményen. Próbára tehetjük a szerencsénket, ami azt jelenti, hogy
                ismét dobnunk kell mindkét dobókockával. Ha a kapott számok összege
                kisebb vagy egyenlő, mint a jelenlegi \stat{szerencse pontjaink} száma,
                akkor vagyunk szerencsések, ellenkező esetben
                balszerencsések.
                \begin{compactitem}[--]
                  \item Ha mi sebeztünk és szerencsénk volt, az azt jelenti, hogy további sebzést
                    tudtunk bevinni a szörnynek, így még 2 pontot levonhatunk tőle.
                  \item Ha mi sebződtünk és szerencsétlenek voltunk, akkor a szörnynek sikerült
                    még egy csapást bevinni, ezért mi még 2 pontot vesztünk.
                  \item Ha mi sebeztünk, de szerencsétlenül jártunk, akkor egy pontot
                    visszakap az ellenfél.
                  \item Ha sebzést kaptunk és szerencsétlenek voltunk, akkor még egy
                    pontot elveszítünk.
                \end{compactitem} 
            \end{itemize}
            Minden esetben, amikor igénybe vesszük a
            szerencsénket, akkor a szerencsepontjaink száma egyel csökken.
          \item A kör végén ellenőriznünk kell, hogy nem csökkent-e le a mi vagy az
            ellenfelünk \stat{életerő pontja} 0-ra. Abban az esetben ha igen, akkor vagy
            legyőztük a szörnyet, és folytathatjuk tovább kalandunkat, vagy ha
            az ellenfél győzőtt le minket, akkor elölről kell kezdeni az egész
            kalandot az első oldaltól.
        \end{enumerate} 
      \subsection{Szerencse}
        Nemcsak csatában találkozhatunk azzal a kifejezéssel, hogy
        \emph{Tedd próbára a szerencsédet!}. Ekkor nem mi döntünk
        sorsunk alakulásáról, hanem a szerencse befolyásolja kalandunk
        további menetét. Hasonlóan, mint a csatában dobnunk kell a két
        kockával egyszerre. Ha az összegük kevesebb vagy egyenlő, mint
        a \stat{szerencse pontjaink} száma, akkor szerencsések
        vagyunk. Ha azonban nagyobb, akkor szerencsétlenek
        vagyunk. Ennek fényében kell tovább folytatni a
        kalandunkat. Minden szerencsepróba után a szerencse pontjaink
        száma egyel csökken.
    
      \section{Játék menete}
        A kaland az 1. oldalon kezdődik és minden oldal végén választani kell,
        hogy merre szeretnénk tovább haladni. Van, hogy csak két út áll
        előttünk, de van, hogy sok lehetőség közül választhatunk. Cél, hogy
        eljussunk a 400. oldalra (ahol a győzelem vár ránk), melynek eléréséhez
        különböző megmérettetések, harcok, próbák, észjátékok várnak ránk.
        Először kocka dobásokkal meghatározzuk a tulajdonságainkat, majd
        elolvashatjuk a háttértörténetet, amelyből kiderül, hogy a hősnek,
        akinek kalandjait mi választjuk meg, mi a célja. Honnan indul és
        hová akar érkezni.
    \section{Program indítása, használata}
      TODO
      
\chap{Fejlesztői dokumentáció} A nagyprogramot \proglang{Haskell}
nyelven implementáltam. Alapvetően két részből áll maga a program, egy
motorból és a megjelenítésből. A program bemenete egy \proglang{xml}
fájl. A motor beolvassa ezt a fájlt, átfordítja egy saját nyelvre, ott
elvégzi a műveleteket, állapotváltozásokat, majd előállít egy
kimenetet. Ezt a kimenetet bárhogy fel lehet dolgozni, én két féle
feldolgozást implementáltam. Az egyik egy graf{}ikus felhasználói
felület, a másik egy webes megvalósítás. Ebből látszik, hogy teljesen
elkülönül egymástól a motor és a megjelenítés.

  \section{Dokumentáció a történetíráshoz}
    Mivel a program egy \proglang{xml} fájlt olvas be, ezért ott kell
    leírni, hogy a könyv különböző sorai milyen műveleteket
    takarnak. Maga az olvasó is találhat ki történeteket. Ehhez az
    \proglang{xml} szerkezetének ismerete szükséges.

    \begin{itemize}
      \item <{\color{Tag}book}> : Gyökér jelölő.

      \item <{\color{Tag}page} {\color{Attr}id}={\color{Value}''400''}
        {\color{Attr}win}={\color{Value}''1''}> : A könyv oldalakból
        tevődik össze. Minden új oldal egy ilyen jelölőbe
        kerül. Attribútumai: \attr{id}, \attr{death}, \attr{win}.  Az
        \attr{id} attribútum kötelező, ez mondja meg, hogy hányadik
        oldalon vagyunk. A másik két attribútum csak különleges
        helyzetekben szerepel. A \attr{death} attribútum azt az
        információt hordozza magában, hogy az oldal, amire érkeztünk
        olyan oldal, ahonnan nincs tovább,vagyis a hős kalandja itt
        véget ér, elbukta küldetését. Hasonlóan a \attr{win}
        attribútum jelentése, hogy a hős elérkezett céljához, tehát
        győzött. A nyitó csukó jelölők közé további jelölők jönnek.
    
      \item <{\color{Tag}p}> : bekezdést jelöl, tehát ennek a
        segítségével tudjuk a szöveget tagolni, hogy átláthatóbb,
        olvashatóbb legyen. Így a nyitó és záró jelölő közé főként
        szöveg kerül, de lehetnek további jelölők is.
        
      \item <{\color{Tag}goto}
        {\color{Attr}ref}={\color{Value}''2''}/> : Az adott oldal
        befejeztével hová tudunk továbbmenni. A \attr{ref} attribútum
        mondja meg, hogy hol folytathatjuk kalandunkat. Ennek a
        jelölőnek a hatása, hogy az oldalra kikerül a \emph{,,lapozz a
          2. oldalra''} szöveg.
    
      \item <{\color{Tag}Goto}
        {\color{Attr}ref}={\color{Value}''2''}/> : Hasonló az
        előzőhöz. A különbség köztük, hogy míg az előző jelölőt egy
        mondat közepén vagy végén használjuk, addig ez egy új mondatot
        állít elő a kimenetben. \emph{,,Lapozz a 2. oldalra.''}
    
      \item <{\color{Tag}text}> <{\color{Tag}/text}>: Szöveg
        kiiratása. Nem új bekezdésben folytatódik a szöveg, hanem az
        adott sorban.
    
      \item <{\color{Tag}goto-lucky}
        {\color{Attr}refYes}={\color{Value}''135''}
        {\color{Attr}refNo}={\color{Value}''273''}/> : Tedd próbára a
        szerencsédet!. Attribútumai: \attr{refYes}, \attr{refNo}. Két
        kockadobásra van szükségünk. Amennyiben a kockadobások
        szerencsét hoztak nekünk, akkor a \attr{refYes} attribútum
        által meghatározott oldalra megyünk, ellenkező esetben a
        \attr{refNo} attribútum mondja meg, hogy kalandunk merre
        folytatódjon.
    
      \item <{\color{Tag}dice} {\color{Attr}name}={\color{Value}''d''}/>
        : Kockadobást. A name attribútum a kocka nevét jelöli, mert
        esetenként több kockadobásra is szükségünk van. Tehát meg kell
        tudnunk különböztetni őket.
    
      \item <{\color{Tag}var} {\color{Attr}ref}={\color{Value}''d''}/> :
        Változóhivatkozás. A d változó értékét adja vissza.
    
      \item <{\color{Tag}intlit}
        {\color{Attr}value}={\color{Value}''5''}/> : Számliterál. A
        \attr{value} attribútumban kapott szám az értéke.
    
      \item <{\color{Tag}lt}> <{\color{Tag}/lt}> : <. (Kisebb.) Binér
        reláció, tehát két argumentuma van. Igaz értékkel tér vissza,
        ha az első argumentum kisebb, mint a második. Az argumentumai
        lehetnek változóhivatkozások, számliterálok.
    
      \item <{\color{Tag}le}> <{\color{Tag}/le}> : <=. (Kisebb vagy
        egyenlő.) Binér reláció. Igaz az értéke, ha az első argumentum
        kisebb vagy egyenlő, mint a második.
    
      \item <{\color{Tag}gt}> <{\color{Tag}/gt}> : >. (Nagyobb.) Binér
        reláció. Igaz, ha az első argumentum nagyobb, mint a második.
    
      \item <{\color{Tag}ge}> <{\color{Tag}/gt}> : >=. (Nagyobb vagy
        egyenlő.) Binér reláció. Igaz, ha az első argumentuma nagyobb
        vagy egyenlő, mint a második.
    
      \item <{\color{Tag}eq}> <{\color{Tag}/eq}> : ==. (Egyenlő.)
        Binér reláció. Igaz, ha a két argumentuma egyenlő.
    
      \item <{\color{Tag}and}> <{\color{Tag}/and}> : És. Argumentumai
        logikai értékek. Igaz értékkel tér vissza, ha minden
        argumentumának értéke igaz, különben hamis értekkel.
    
      \item <{\color{Tag}or}> <{\color{Tag}/or}> : Vagy. Argumentumai
        logikai értékek. Igaz értékkel tér vissza, ha van olyan
        argumentuma, amelynek értéke igaz. Különben hamissal.
    
      \item <{\color{Tag}not}> <{\color{Tag}/not}> : Negáció. Egy
        argumentuma van, mely egy logikai érték. Igazhoz hamisat,
        hamishoz igazat rendel.
    
      \item <{\color{Tag}cond}> <{\color{Tag}/cond}> : Feltétel. Három
        argumentuma van. Amennyiben az első argumentum értéke igaz,
        akkor a második argumentumra kerül a vezérlés, különben a
        másodikra.
    
      \item <{\color{Tag}if}> <{\color{Tag}/if}> : Elágazás. Van egy
        feltétele, melynek teljesülése esetén az igaz ágra kerül a
        vezérlés, különben a hamis ágra. Lehetséges az is, hogy nincs
        hamis ág. Ebben az esetben, ha a feltétel nem teljesül, akkor
        a következő jelölővel folytatódik a történet.  pl:
    
      \item <{\color{Tag}take}
        {\color{Attr}item}={\color{Value}''item''}/> : Eszköz
        felvétele. Kalandunk során számon kell tartanunk, hogy milyen
        eszközökre tettünk szert. Ezekre leginkább a későbbiekben lesz
        szükségünk. Elérkezhetünk egy olyan megmérettetéshez, melynek
        sikeressége az adott tárgy birtoklásán múlhat.
    
      \item <{\color{Tag}drop}
        {\color{Attr}item}={\color{Value}''item''}/> : Tárgy eldobása,
        felhasználása. Ha felhasználunk egy korábban megszerzett
        tárgyat, de az már a későbbiekben nem marad nálunk, akkor
        használjuk ezt a jelölőt.
    
      \item <{\color{Tag}inc}
        {\color{Attr}counter}={\color{Value}''item''}/> : Lehet az a
        feladatunk a kaland során, hogy egy adott tárgyból össze kell
        gyűjtenünk többet. Ennek a segítségével tudjuk az adott eszköz
        számát nyilvántartanunk, tehát ha találunk egy újabbat, akkor
        csak annyit jegyzünk meg, hogy találtunk még egyet.
    
      \item <{\color{Tag}dec}
        {\color{Attr}counter}={\color{Value}''item''}/> : Amikor a
        gyűjtés során elveszítünk vagy felhasználunk egy tárgyat.
    
      \item <{\color{Tag}set-flag}
        {\color{Attr}flag}={\color{Value}''flag''}/> : Szükségünk lehet
        arra is, hogy megjegyezzünk olyan dolgokat, hogy már
        megtörtént velünk egy bizonyos dolog. A későbbiekben ez
        befolyásolhatja utunkat.
    
      \item <{\color{Tag}heal}
        {\color{Attr}stat}={\color{Value}''stat}''> <{\color{Tag}intlit}
        {\color{Attr}value}={\color{Value}''1''}> <{\color{Tag}/heal}> :
        A stat attribútum által meghatározott jellemzőnk
        növelése. Annyival nő az értéke, amennyi a nyitó és záró
        jelölő közötti érték.
    
      \item <{\color{Tag}damage}
        {\color{Attr}stat}={\color{Value}''health''}>
        <{\color{Tag}intlit} {\color{Attr}value}={\color{Value}''1''}>
        <{\color{Tag}/damage}> : A stat attribútum által meghatározott
        jellemzőnk csökkenése. Annyival lesz kevesebb az értéke,
        amennyi a nyitó es záró jelölő közötti érték.
    
      \item <{\color{Tag}counter}
        {\color{Attr}name}={\color{Value}''item''}/> : A \attr{name}
        attribútumban meghatározott tárgy számát adja meg, hogy
        mennyit sikerült összegyűjtenünk belőle.

      \item <{\color{Tag}clear}
        {\color{Attr}counter}={\color{Value}''item''}/> : A gyűjtött
        tárgyat elveszítjük, vagyis a száma 0 lesz.
        
      \item <{\color{Tag}carry}
        {\color{Attr}item}={\color{Value}''item''}/> : Rendelkezünk-e az
        adott tárggyal.
    
      \item <{\color{Tag}plus}> <{\color{Tag}/plus}> : Két argumentuma
        van, visszaadja ezeknek az attribútumoknak az összegét.
    
      \item <{\color{Tag}score}
        {\color{Attr}stat}={\color{Value}''stat''}/> : A \attr{stat}
        attribútumban meghatározott jellemzőnk lekérdezése.
    
      \item <{\color{Tag}f{}ight}> <{\color{Tag}/f{}ight}> : Harc egy
        mesebeli lénnyel.
    
      \item <{\color{Tag}enemy}
        {\color{Attr}agility}={\color{Value}''10''}
        {\color{Attr}health}={\color{Value}''8''}> </enemy> : Az
        ellenfél, akivel harcolnunk kell. Az attribútumai megadják a
        lény \stat{ügyességét} és \stat{ételerejét}. A jelölők közti
        szöveg, pedig a nevét.
    \end{itemize}
  \section{Fejlesztői dokumentáció}
    \subsection{Motor}
    \subsection{Megjelenítés}
\end{document}
